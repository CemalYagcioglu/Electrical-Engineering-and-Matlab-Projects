%% IMPORTANT: Once working, run latex 3 times to get listoffigures to work

%% Be sure to check spelling!

%% Put **your** name and the proper due date in place

%% Note put your own file names in the appendix
%% Copy the relevant code as many times as needed for all files

%% Note that the \epsfig command is currently commented out

\documentclass{article}
\usepackage{amsmath}    % load AMS-Math package
\usepackage{epsfig}     % allows PostScript files
\usepackage{listings}   % allows lstlisting environment
\usepackage{moreverb}   % allows listinginput environment
\usepackage{vmargin}    % allows better margins
\setpapersize{USletter} % sets the paper size
\setmarginsrb{1in}{0.5in}{1in}{0.2in}{12pt}{11mm}{0pt}{11mm} %sets margins 
\begin{document}
\begin{center}
\rule{6.5in}{0.5mm}\\~\\
{\bf \large EGR 103L - Fall 2016}\\~\\
{\huge \bf Laboratory 7 - Linear Algebra}\\~\\
CEMAL YAGCIOGLU (cy111)\\
Lab Section 5D, Wednesday 11.45AM - 2.35PM\\
30 OCTOBER, 2016\\~\\
{\small  I understand and have adhered to all the tenets of the Duke
  Community Standard in completing every part of this assignment.  I
  understand that a violation of any part of the Standard on any part
  of this assignment can result in failure of this assignment, failure
  of this course, and/or suspension from Duke University.} 
\rule{6.5in}{0.5mm}\\
\end{center}
\tableofcontents
\listoffigures
\pagebreak
\section{Palm Problem 8.1}
\renewcommand{\arraystretch}{1.5}
\begin{center}
\begin{tabular}{|c||c|c|c|}\hline
Part & x & y & z \\ \hline \hline
$a$ &  &  & N/A\\ \hline
$b$ &  &  & N/A \\ \hline
$c$ &  &  & \\ \hline
$d$ &  &  & \\ \hline
\end{tabular}
\end{center}

\section{Based on Chapra Problem 8.3}
% Equations in matrix form
% Solutions for this system (x = vector)
% Transpose and inverse
% Condition numbers
% Meaning of condition numbers

\section{Based on Chapra Problem 8.10}
% Equations in matrix form
% Sentence introducing the following results: 
\listinginput[1]{1}{TrussData.txt}

\section{Palm 8.5(b)}
% Discussion of whether there is a c value with easily predicted solutions
% Discussion of why there are 201 points

\section{Based on Palm 8.9}
% Equations in matrix form
\ begin { align *}
\ begin { bmatrix }
3 & -1 & -1 & 0 \\
-1 & 2 & 0 & -1 \\
-1 & 0 & 2 & -1 \\
0 & -1 & -1 & 3 \\
\ end { bmatrix }
\ begin { Bmatrix }
$T_1~ \\ T_2 \\ T_3 \\ T_4$
\ end { Bmatrix }&=
\ begin { Bmatrix }
$T_a \\ 0 \\ 0 \\ T_b$
\ end { Bmatrix }
% Sentence introducing the following results:
\\Temperature data for $T_a$=120 degree Celcius and $T_b$=20 degree Celcius:
\listinginput[1]{1}{TempData.txt}

\section{Based on Palm 8.16(a)}
% Sentence introducing the table of results
Calculated coefficients of for the set of points are in the table below.

% Table of results.
\begin{tabular}{|c|c|c|c|}
\hline
Points & a & b & c\\\hline
(1,4), (4, 73), (5, 120) & 6.00e+00 & -7.00e+00 & 5.00e+00\\
(1,4), (4, -73), (5, 120) & 5.47e+01 & -2.99e+02 & 2.48e+02\\
(1,4), (4, 73), (4, 120) & \bf{N/A} & \bf{N/A} & \bf{N/A}\\
(1,4), (4, 73), (5, -120) & -5.40e+01 & 2.93e+02 & -2.35e+02\\\hline
\end{tabular}
\end{center}

\pagebreak
\appendix
\section{Codes and Output}
% Put the name of your file in the subsection name 
% and the listinginput input
% Be sure to include the community standard in codes!
% Add \pagebreaks if they make sense
% Make as many copies as you need

\subsection{PROBLEM}
\listinginput[1]{1}{File.m}

\pagebreak
\section{Figures}

\begin{figure}[htb!]
\begin{center}
%\epsfig{file=FIGURE.eps, width=3.in}
\caption{CAPTION}
\end{center}
\end{figure}

\end{document}


